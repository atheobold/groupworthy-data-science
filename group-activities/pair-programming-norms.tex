% Options for packages loaded elsewhere
\PassOptionsToPackage{unicode}{hyperref}
\PassOptionsToPackage{hyphens}{url}
\PassOptionsToPackage{dvipsnames,svgnames,x11names}{xcolor}
%
\documentclass[
  letterpaper,
  DIV=11,
  numbers=noendperiod]{scrartcl}

\usepackage{amsmath,amssymb}
\usepackage{iftex}
\ifPDFTeX
  \usepackage[T1]{fontenc}
  \usepackage[utf8]{inputenc}
  \usepackage{textcomp} % provide euro and other symbols
\else % if luatex or xetex
  \usepackage{unicode-math}
  \defaultfontfeatures{Scale=MatchLowercase}
  \defaultfontfeatures[\rmfamily]{Ligatures=TeX,Scale=1}
\fi
\usepackage{lmodern}
\ifPDFTeX\else  
    % xetex/luatex font selection
\fi
% Use upquote if available, for straight quotes in verbatim environments
\IfFileExists{upquote.sty}{\usepackage{upquote}}{}
\IfFileExists{microtype.sty}{% use microtype if available
  \usepackage[]{microtype}
  \UseMicrotypeSet[protrusion]{basicmath} % disable protrusion for tt fonts
}{}
\makeatletter
\@ifundefined{KOMAClassName}{% if non-KOMA class
  \IfFileExists{parskip.sty}{%
    \usepackage{parskip}
  }{% else
    \setlength{\parindent}{0pt}
    \setlength{\parskip}{6pt plus 2pt minus 1pt}}
}{% if KOMA class
  \KOMAoptions{parskip=half}}
\makeatother
\usepackage{xcolor}
\setlength{\emergencystretch}{3em} % prevent overfull lines
\setcounter{secnumdepth}{-\maxdimen} % remove section numbering
% Make \paragraph and \subparagraph free-standing
\makeatletter
\ifx\paragraph\undefined\else
  \let\oldparagraph\paragraph
  \renewcommand{\paragraph}{
    \@ifstar
      \xxxParagraphStar
      \xxxParagraphNoStar
  }
  \newcommand{\xxxParagraphStar}[1]{\oldparagraph*{#1}\mbox{}}
  \newcommand{\xxxParagraphNoStar}[1]{\oldparagraph{#1}\mbox{}}
\fi
\ifx\subparagraph\undefined\else
  \let\oldsubparagraph\subparagraph
  \renewcommand{\subparagraph}{
    \@ifstar
      \xxxSubParagraphStar
      \xxxSubParagraphNoStar
  }
  \newcommand{\xxxSubParagraphStar}[1]{\oldsubparagraph*{#1}\mbox{}}
  \newcommand{\xxxSubParagraphNoStar}[1]{\oldsubparagraph{#1}\mbox{}}
\fi
\makeatother


\providecommand{\tightlist}{%
  \setlength{\itemsep}{0pt}\setlength{\parskip}{0pt}}\usepackage{longtable,booktabs,array}
\usepackage{calc} % for calculating minipage widths
% Correct order of tables after \paragraph or \subparagraph
\usepackage{etoolbox}
\makeatletter
\patchcmd\longtable{\par}{\if@noskipsec\mbox{}\fi\par}{}{}
\makeatother
% Allow footnotes in longtable head/foot
\IfFileExists{footnotehyper.sty}{\usepackage{footnotehyper}}{\usepackage{footnote}}
\makesavenoteenv{longtable}
\usepackage{graphicx}
\makeatletter
\def\maxwidth{\ifdim\Gin@nat@width>\linewidth\linewidth\else\Gin@nat@width\fi}
\def\maxheight{\ifdim\Gin@nat@height>\textheight\textheight\else\Gin@nat@height\fi}
\makeatother
% Scale images if necessary, so that they will not overflow the page
% margins by default, and it is still possible to overwrite the defaults
% using explicit options in \includegraphics[width, height, ...]{}
\setkeys{Gin}{width=\maxwidth,height=\maxheight,keepaspectratio}
% Set default figure placement to htbp
\makeatletter
\def\fps@figure{htbp}
\makeatother

\KOMAoption{captions}{tableheading}
\makeatletter
\@ifpackageloaded{caption}{}{\usepackage{caption}}
\AtBeginDocument{%
\ifdefined\contentsname
  \renewcommand*\contentsname{Table of contents}
\else
  \newcommand\contentsname{Table of contents}
\fi
\ifdefined\listfigurename
  \renewcommand*\listfigurename{List of Figures}
\else
  \newcommand\listfigurename{List of Figures}
\fi
\ifdefined\listtablename
  \renewcommand*\listtablename{List of Tables}
\else
  \newcommand\listtablename{List of Tables}
\fi
\ifdefined\figurename
  \renewcommand*\figurename{Figure}
\else
  \newcommand\figurename{Figure}
\fi
\ifdefined\tablename
  \renewcommand*\tablename{Table}
\else
  \newcommand\tablename{Table}
\fi
}
\@ifpackageloaded{float}{}{\usepackage{float}}
\floatstyle{ruled}
\@ifundefined{c@chapter}{\newfloat{codelisting}{h}{lop}}{\newfloat{codelisting}{h}{lop}[chapter]}
\floatname{codelisting}{Listing}
\newcommand*\listoflistings{\listof{codelisting}{List of Listings}}
\makeatother
\makeatletter
\makeatother
\makeatletter
\@ifpackageloaded{caption}{}{\usepackage{caption}}
\@ifpackageloaded{subcaption}{}{\usepackage{subcaption}}
\makeatother

\ifLuaTeX
  \usepackage{selnolig}  % disable illegal ligatures
\fi
\usepackage{bookmark}

\IfFileExists{xurl.sty}{\usepackage{xurl}}{} % add URL line breaks if available
\urlstyle{same} % disable monospaced font for URLs
\hypersetup{
  pdftitle={Pair Programming Roles},
  colorlinks=true,
  linkcolor={blue},
  filecolor={Maroon},
  citecolor={Blue},
  urlcolor={Blue},
  pdfcreator={LaTeX via pandoc}}


\title{Pair Programming Roles}
\author{}
\date{}

\begin{document}
\maketitle


During our in-class activities, you will be paired with another student.
When completing the activity, you will rotate between the following
roles every 2-3 minutes:

\includegraphics{cats-programming.jpg}

\textbf{Coder} -- \emph{Proposes solution strategies}

\begin{itemize}
\tightlist
\item
  Reads out instructions or prompts
\item
  Directs the Developer what to type in the Quarto document in RStudio
\item
  Talks with Developer about ideas related to ways they could solve the
  problem.
\item
  \textbf{Does not} ask the Developer how they would solve the problem.
\item
  Manages resources (e.g., cheatsheets, textbook) for aiding in solution
  strategy.
\item
  \textbf{Does not} ask the developer what functions / tools they should
  use.
\item
  Revises solution strategy based on evaluation / debugging
  conversations with the Developer.
\item
  \textbf{Does not} ask the Developer to debug the code.
\item
  Talks with the Developer about what comments to make in the Quarto
  document to help the group understand the code later.
\end{itemize}

\textbf{Developer} -- \emph{Evaluates solution strategy}

\begin{itemize}
\tightlist
\item
  Reads prompt and ensures Coder understands what the problem is asking.
\item
  If neither party understands what the problem is asking, raises hand
  to ask a group question to Dr.~Theobold.
\item
  Types the code specified by the Coder into the Quarto document in
  RStudio
\item
  Listens carefully, asks the Coder to repeat statements if needed, or
  to slow down.
\item
  Encourages the Coder to vocalize their thinking.
\item
  Asks the Coder clarifying questions.
\item
  Checks for accuracy by asking the solution to be restated for clarity.
\item
  \textbf{Does not} give hints to the Coder for how to solve the
  problem.
\item
  \textbf{Does not} solve the problem themselves.
\item
  Runs the code provided by the Coder.
\item
  If code does not run or has errors, talks with Coder about how to
  debug the proposed solution strategy.
\item
  \textbf{Does not} tell the Coder how to correct an error.
\item
  Once code runs, evaluates the output relative to the prompt to ensure
  the output addresses the question.
\item
  Talks with the Coder about what comments to make in the Quarto
  document to help the group understand the code later
\end{itemize}

\subsection{Group Norms}\label{group-norms}

\begin{enumerate}
\def\labelenumi{\arabic{enumi}.}
\tightlist
\item
  Think and work together. Do not divide the work.
\item
  You are smarter together.
\item
  Be open minded.
\item
  No cross-talk with other groups.
\item
  Communicate with each other!
\end{enumerate}

\subsection{Completing the Task}\label{completing-the-task}

Working with your partner, complete the Practice Activity in the Quarto
document provided. In your roles---Coder and Developer---use the prompts
below to help guide the completion of your activity.

\textbf{Coder}

I am looking for\ldots{}\\
I am confused by\ldots{}\\
I notice\ldots{}\\
What might be true is\ldots{}\\
What's important is\ldots{}\\
I predict\ldots{}\\
This reminds me of\ldots{}\\
What if we\ldots{}\\
We could try\ldots{}\\
I am thinking about\ldots{}\\
A line of code I could write is\ldots{}

\textbf{Developer}

Do you understand what we need to do?\\
What's the question we have for Dr.~Theobold? Can we answer the question
ourselves?\\
What are you focusing on?\\
What are you thinking now?\\
Could you tell me more?\\
What are you doing (or writing) now? Please elaborate.\\
I can't follow that, run that by me again.\\
What other sources of information do we need?\\
Which words should we look up?\\
What else do we need to complete this problem?

\subsection{Once You're Finished}\label{once-youre-finished}

At the end of the task, your group will have one completed Quarto
document and one rendered HTML, containing your groups' worked-out
solutions and justifications. \textbf{Everyone} must take turns writing
the final product (as described above) and \textbf{everyone} must be
able to explain every line of code in your final document.




\end{document}
